\documentclass[dvipdfmx,autodetect-engine,10pt]{jsarticle}% autodetect-engine で pLaTeX / upLaTeX を自動判定
\setlength{\columnsep}{3zw}
\usepackage[dvipdfmx]{graphicx}
\usepackage{amsmath,amssymb}

\title{矩形マルチピンホールSPECTシステムの提案}
\date{2019年9月17日}

\begin{document}


\maketitle
\section{はじめに}
SPECT(Single Photon Emission Computed Tomography)とは放射性同位元素(RI:Radio Isotope)を用いた放射性医薬品を体内に投与することによって,放射性医薬品から出る微量な放射線(γ線)をさまざなま方向から測定し,断層画像にする方法である.

SPECTによる測定において,γ線を収集する方向を一定にするために,コリメータと呼ばれる装置を用いる.コリメータのピンホールは本来円形をしているが,ピンホールの形を矩形にした,マルチ矩形ピンホールSPECTシステムの開発を研究テーマとして研究を行なっている.通常ピンホールは円形をしているが.有効視野の範囲外になってしまう部分,すなわち,検出器の中に,検出に使用しない領域が発生してしまう.検出に使用しない領域を小さくするために,ピンホールを矩形した,矩形マルチピンホールSPECTシステムを提案する.



\end{document}
