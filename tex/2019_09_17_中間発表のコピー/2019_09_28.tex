\documentclass[dvipdfmx,autodetect-engine,twocolumn,10pt]{jsarticle}% autodetect-engine で pLaTeX / upLaTeX を自動判定
\setlength{\columnsep}{3zw}
\usepackage[dvipdfmx]{graphicx}
\usepackage{amsmath,amssymb}

\title{矩形マルチピンホールSPECTシステムの提案}
\author{法政大学理工学部 応用情報工学科 4年 16X3128 馬場俊弥}
\date{2019年9月28日}

\begin{document}


\maketitle
\section{はじめに}
% \abstract{
SPECT(Single Photon Emission Computed Tomography)とは放射性同位元素(RI:Radio Isotope)を用いた放射性医薬品を体内に投与することによって,放射性医薬品から出る微量な放射線(γ線)をさまざなま方向から測定し,断層画像にする方法である.
%
% この研究を進めるにあたり,現在,マルチピンホールコリメータを用いた,モンテカルロシミュレーションの実装を行なっている.今回はその過程で実装を行なった投影データ取得,感度補正について紹介する.また,ピンホールの形を矩形にしたものを実装したため,それも合わせて紹介する.
% }



\section{研究概要}
SPECTによる測定において,γ線を収集する方向を一定にするために,コリメータと呼ばれる装置を用いる.コリメータのピンホールは本来円形をしているが,ピンホールの形を矩形にした,マルチ矩形ピンホールSPECTシステムの開発を研究テーマとして研究を行なっている.通常ピンホールは円形をしているが.有効視野の範囲外になってしまう部分,すなわち,検出器の中に,検出に使用しない領域が発生してしまう.検出に使用しない領域を小さくするために,ピンホールを矩形した,矩形マルチピンホールSPECTシステムを提案する.

円形ピンホール,矩形ピンホールの有効視野の例を図\ref{fov}に示す.
\begin{figure}[htbp]
  \begin{center}
    \begin{tabular}{c}

      % 1
      \begin{minipage}{0.5\hsize}
        \begin{center}
          \includegraphics[clip, width=3.5cm]{./file/fov_11pinhole_5mm_int_512-256.png}
          \hspace{1cm} \small{(a)円型ピンホール}
        \end{center}
      \end{minipage}

      % 2
      \begin{minipage}{0.5\hsize}
        \begin{center}
          \includegraphics[clip, width=3.5cm]{./file/fov_11pinhole_square_5mm_int_512-256.png}
          \hspace{1.6cm} \small{(b)矩形ピンホール}
        \end{center}
      \end{minipage}

    \end{tabular}
  \caption{ピンホールの有効視野}
  \label{fov}
  \end{center}
\end{figure}

\newpage
ピンホールの配置や傾きを変えることによって,検出に使用しない領域を,最小限にすることにより,検出効率をあげることが研究の目的である.その際に,矩形の方が,検出器の角の周辺を有効視野にすることが容易であり,有効視野同士の重なりを少なくすることができる.

\section{どのように研究を進めていくか}
\begin{enumerate}
  \item 再構成ができることが確認されている条件で,再構成するための処理を実装
  \item 矩形ピンホールで再構成を行う
  \item 矩形ピンホールを用いたことによる問題を改善するためのソフトウェアの処理を研究

\end{enumerate}

\section{データ収集方法}
データ収集にはモンテカルロ法を用いた.モンテカルロ法とは,乱数を用いて解析的に解けない問題に対してシミュレーションを繰り返すことで近似解を得る手法のこと.光子の振る舞いを仮定し,相互作用の影響を考慮して現実の状況に近いデータを取得できる.


\section{再構成手法}
再構成には,ML-EM法を用いる.ML-EM法とは,Maximum Likelihood – Expectation Maximizationの略.取得した投影データと再構成画像との比較を繰り返し行うことにより,再構成画像を更新する方法である.
\[
  \lambda_j^{k+1}=\frac{\lambda_j^{(k)}}{\sum_{i=1}^{n} C_{ij}}\sum_{i=1}^{n} \frac{y_iC_{ij}}{\sum_{j^{\prime}=1}^{m}C_{ij^{\prime}\lambda_{j^{\prime}^{(k)}}}}
\]


\section{丸型ピンホールのシミュレーション}
\subsection{シミュレーション条件}
シミュレーション条件を表\ref{simu_2d}に示す.
\begin{table}[htbp]
  \begin{center}
    \caption{シミュレーション条件}
    \label{simu_2d}
    \small
    \scalebox{0.82}[0.9]
    {
      \begin{tabular}{|c|c|} \hline
        ファントム & Sheppファントム \\ \hline
        回転半径 & 17 $cm$ \\ \hline
        画像サイズ & 128×128×128 $voxels$ \\ \hline
        画像のボクセルサイズ & 0.2×0.2×0.2 $cm^3$ \\ \hline
        コリメータから検出器までの距離 & 7.6 $cm$ \\ \hline
        検出のサイズ & 512×256 $pixels$ \\ \hline
        検出のピクセルサイズ & 0.08×0.08 $cm^2$ \\ \hline
        検出器の数 & 180 \\ \hline
      \end{tabular}
    }
  \end{center}
\end{table}

ピンホールのジオメトリを図\ref{pinhole}に示す.
\begin{figure}[htbp]
  \begin{center}
    \includegraphics[width=6.5cm]{./file/geometry.png}\\
    \caption{ジオメトリ}
    \label{pinhole}
  \end{center}
\end{figure}

使用した原画像とそのプロファイルを図\ref{original_img}に示す.
\begin{figure}[htbp]
  \begin{center}
    \begin{tabular}{c}

      % 1
      \begin{minipage}{0.5\hsize}
        \begin{center}
          \includegraphics[clip, width=2.5cm]{./file/Shepp_float_64-64-64.png}
        \end{center}
      \end{minipage}

      % 2
      \begin{minipage}{0.5\hsize}
        \begin{center}
          \includegraphics[clip, width=3.5cm]{./file/Shepp_profile.png}
        \end{center}
      \end{minipage}

    \end{tabular}
  \caption{(左)原画像,(右)プロファイル}
  \label{original_img}
  \end{center}
\end{figure}


\subsection{結果}
\subsubsection{データ取得}
モンテカルロ法を用いた光子輸送を用いて取得した投影画像を図\ref{proj}に示す.
\begin{figure}[htbp]
  \begin{center}
    \includegraphics[width=7cm]{./file/project.png}\\
    \caption{投影画像}
    \label{proj}
  \end{center}
\end{figure}

\subsubsection{再構成}
再構成画像とそのプロファイルを図\ref{reconst}に示す.
\begin{figure}[htbp]
  \begin{center}
    \begin{tabular}{c}
      \begin{minipage}{0.5\hsize}
        \begin{center}
          \includegraphics[clip, width=2.5cm]{./file/ML-EM100_float_64-64-64.png}
        \end{center}
      \end{minipage}
      \begin{minipage}{0.5\hsize}
        \begin{center}
          \includegraphics[clip, width=3.5cm]{./file/treconst_profile.png}
        \end{center}
      \end{minipage}
    \end{tabular}
  \caption{(左)再構成画像,(右)プロファイル}
  \label{reconst}
  \end{center}
\end{figure}
\section{まとめと今後の展望}
モンテカルロ法を用いて光子の動きを再現することにより,SPECT処理をシミュレーションすることができるようになった.
今後は,ピンホールを矩形にした場合にどのような問題が発生するか調べる.また,矩形にしたことによって生じた問題を改善するための手法を研究する.

\begin{thebibliography}{9}
  \bibitem{Metzler} S.D. Metzler, S.C. Moore, and M.-A. Park,“Design of a New Small-Animal SPECT System Based on Rectangular Pinhole Aperture.”
  \bibitem{Ogawa} K. Ogawa, S. Takahashi and Y. Satori "Description of an object in Monte Carlo simulations,"
  \bibitem{Fujishiro} Yohei Fujishiro, Kazumi Murata, Nobutoku Motomura, Koichi Ogawa "List Mode Image Reconstruction With a Multi-pinhole Triple Head SPECT System"
\end{thebibliography}




\end{document}
