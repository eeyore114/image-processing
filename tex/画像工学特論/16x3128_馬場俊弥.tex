\documentclass[dvipdfmx,autodetect-engine,twocolumn,10pt]{jsarticle}% autodetect-engine で pLaTeX / upLaTeX を自動判定
\setlength{\columnsep}{3zw}
\usepackage[dvipdfmx]{graphicx}
\usepackage{amsmath,amssymb}

\title{マルチピンホールコリメータの実装}
\author{法政大学理工学部 応用情報工学科 4年 16X3128 馬場俊弥}
\date{\today}

\begin{document}

\maketitle
\section{はじめに}
SPECT(Single Photon Emission Computed Tomography)とは放射性同位元素(RI:Radio Isotope)を用いた放射性医薬品を体内に投与することによって,放射性医薬品から出る微量な放射線(γ線)をさまざなま方向から測定し,断層画像にする方法である.

SPECTによる測定において,γ線を収集する方向を一定にするために,コリメータと呼ばれる装置を用いる.コリメータにはシングルピンホールコリメータ,マルチピンホールコリメータ,パラレルホールコリメータなどがある.コリメータのピンホールは本来円形をしているが,ピンホールの形を矩形にした,マルチ矩形ピンホールSPECTの開発を研究テーマとして研究を行なっている.この研究を進めるにあたり,今回は,マルチピンホールコリメータの実装を行った.ray tracingで投影データを取得し,再構成を行なった結果を2次元,3次元でそれぞれ紹介する.

\newpage

\section{2次元マルチピンホールコリメータ}

\subsection{実装方法}
ピンホールを複数にする場合,それぞれのピンホールから物体全体を捉えるためには,それぞれのピンホールに傾きをつける必要がある.
\begin{figure}[htbp]
  \begin{center}
    \includegraphics[width=7.5cm]{./file/gradient.png}\\
    \caption{左:傾き無し,右:傾き有り}
    \label{gradient}
  \end{center}
\end{figure}

\subsection{シミュレーション条件}
シミュレーション条件を表\ref{simu_2d}に示す.

\begin{table}[htbp]
  \begin{center}
    \caption{シミュレーション条件}
    \label{simu_2d}
    \small
    \scalebox{0.82}[0.9]
    {
      \begin{tabular}{|c|c|} \hline
        ファントム & Sheppファントム \\ \hline
        長径,短径 & 24.4 cm, 18.4 $cm$ \\ \hline
        回転半径 & 25 $cm$ \\ \hline
        画像サイズ & 128×128 $pixel$ \\ \hline
        画像のピクセルサイズ & 0.2×0.2 $cm^2$ \\ \hline
        コリメータから検出器までの距離 & 7.5 $cm$ \\ \hline
        検出のサイズ & 512 $pixel$ \\ \hline
        検出のピクセルサイズ & 0.08 $cm$ \\ \hline
        投影数 & 180 \\ \hline
        コリメータ & 無限小(30度以内は通過) \\ \hline
        コリメータの位置 & 左から x = - 9, 0, 9 \\ \hline
      \end{tabular}
    }
  \end{center}
\end{table}

\subsection{結果}
投影画像を図\ref{proj_2d},再構成結果を図\ref{reconst_2d},プロファイルを図\ref{profile_2d}にそれぞれ示す.

\begin{figure}[htbp]
  \begin{center}
    \includegraphics[width=7.5cm]{./file/detector_float_512-180.png}\\
    \caption{投影画像}
    \label{proj_2d}
  \end{center}
\end{figure}

\begin{figure}[htbp]
  \begin{center}
    \includegraphics[width=3cm]{./file/ML-EM100_float_128-128.png}\\
    \caption{再構成結果(ML-EM 100回)}
    \label{reconst_2d}
  \end{center}
\end{figure}

\begin{figure}[htbp]
  \begin{center}
    \includegraphics[width=6cm]{./file/profile_2d.png}\\
    \caption{プロファイル}
    \label{profile_2d}
  \end{center}
\end{figure}

\newpage


\section{3次元マルチピンホールコリメータ}

\subsection{シミュレーション条件}
シミュレーション条件を表\ref{simu_3d}に示す.

\begin{table}[htbp]
  \begin{center}
    \caption{シミュレーション条件}
    \label{simu_3d}
    \small
    \scalebox{0.82}[0.9]
    {
      \begin{tabular}{|c|c|} \hline
        ファントム & Sheppファントム \\ \hline
        長径,短径 & 24.4 cm, 18.4 $cm$ \\ \hline
        回転半径 & 25 $cm$ \\ \hline
        画像サイズ & 128×128×128 $voxel$ \\ \hline
        画像のピクセルサイズ & 0.2×0.2×0.2 $cm^3$ \\ \hline
        コリメータから検出器までの距離 & 7.0 $cm$ \\ \hline
        検出のサイズ & 512×256 $pixel$ \\ \hline
        検出のピクセルサイズ & 0.08×0.08 $cm^2$ \\ \hline
        投影数 & 180 \\ \hline
        コリメータ & 無限小(30度以内は通過) \\ \hline
      \end{tabular}
    }
  \end{center}
\end{table}

\subsection{結果}
有効視野を図\ref{fov},投影画像を図\ref{proj_3d},再構成結果を図\ref{reconst_3d},プロファイルを図\ref{profile_3d}にそれぞれ示す.

\begin{figure}[htbp]
  \begin{center}
    \includegraphics[width=7.5cm]{./file/fov_float_512-256.png}\\
    \caption{有効視野}
    \label{fov}
  \end{center}
\end{figure}

\begin{figure}[htbp]
  \begin{center}
    \includegraphics[width=7.5cm]{./file/proj_float_512-256-50.png}\\
    \caption{投影画像}
    \label{proj_3d}
  \end{center}
\end{figure}

\begin{figure}[htbp]
  \begin{center}
    \includegraphics[width=3cm]{./file/ML-EM200_float_128-128-128.png}\\
    \caption{再構成結果(ML-EM 100回)}
    \label{reconst_3d}
  \end{center}
\end{figure}

\begin{figure}[htbp]
  \begin{center}
    \includegraphics[width=6cm]{./file/profile_3d.png}\\
    \caption{プロファイル}
    \label{profile_3d}
  \end{center}
\end{figure}

\section{まとめと今後の展望}
3次元のマルチピンホールが想像よりも綺麗な画像にならなかったため,原因を特定する.また,今回はコリメータは厚さ 0 cmの理想的なコリメータで実装を行なったが,ナイフエッジにして実装を行う.その後,以前行なったモンテカルロシミュレーションをマルチピンホールコリメータを用いて行う.

\end{document}
