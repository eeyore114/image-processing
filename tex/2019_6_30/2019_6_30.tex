\documentclass[dvipdfmx,autodetect-engine,twocolumn,10pt]{jsarticle}% autodetect-engine で pLaTeX / upLaTeX を自動判定
\setlength{\columnsep}{3zw}
\usepackage[dvipdfmx]{graphicx}
\usepackage{amsmath,amssymb}

\title{シングルピンホールを用いた
3次元画像のモンテカルロシミュレーション}
\author{法政大学理工学部 応用情報工学科 4年 16X3128 馬場俊弥}
\date{2019年6月30日}

\begin{document}

\maketitle
\section{はじめに}
SPECT(Single Photon Emission Computed Tomography)とは放射性同位元素(RI:Radio Isotope)を用いた放射性医薬品を体内に投与することによって,放射性医薬品から出る微量な放射線(γ線)をさまざなま方向から測定し,断層画像にする方法である.

SPECTによる測定において,γ線を収集する方向を一定にするために,コリメータと呼ばれる装置を用いる.コリメータにはシングルピンホールコリメータ,マルチピンホールコリメータ,パラレルホールコリメータなどがある.コリメータのピンホールは本来円形をしているが,ピンホールの形を矩形にした,マルチ矩形ピンホールSPECTの開発を研究テーマとして研究を行なっている.この研究を進めるにあたり,今回は,モンテカルロシミュレーションを行うにあたり使用した感度補正,吸収補正,そしてサンプル点補間の紹介,また,通常の円形のコリメータを用いて,シングルピンホールコリメータを用いた3次元画像のモンテカルロシミュレーションの結果を紹介する.


\section{補正,補間処理}

\subsection{感度補正}
感度補正とはピンホールコリメータで検出する際に生じる検出感度の不均一を考慮したものである.理想的な直線モデルを$P_{ideal}$,感度分布を$P_{sensitivity}$から感度法制フィルタ$P_{filter}$を作成し,投影データに掛け合わせることで補正を行う.
\[
  P_{filter} = \frac{P_{ideal}}{P_{sensitivity}}
\]

\subsection{吸収補正}
光子が媒質と相互作用を起こすことによって、理想的なカウントよりも減少してしまう. これを考慮するために吸収補正を行う.吸収補正の式を以下に示す.

\[
  N' = N exp(\mu x)
\]

\begin{table}[htbp]
  \begin{center}
    \begin{tabular}{|c|c|} \hline
      $N$ & 発生光子数 \\ \hline
      $N'$ & 検出光子数 \\ \hline
      $\mu$ & 線減衰係数 \\ \hline
      $x$ & 媒質中の移動距離 \\ \hline
    \end{tabular}
  \end{center}
\end{table}

\subsection{サンプル点補間}
検出器のサンプル点から仮想的に伸ばす投影線が,再構成空間上で通過しない箇所がある場合にアーチファクトが生じてしまう.これを防ぐために,検出器のサンプル点を仮想的に増やす処理を行う.

\newpage

\section{3次元画像のモンテカルロシミュレーション}
シミュレーションに使用した原画像を図\ref{original_img}に,シミュレーション条件を表\ref{simu_cond}に示す.

\begin{figure}[htbp]
  \begin{center}
    \includegraphics[width=3.0cm]{./file/single_original.png}\\
    \caption{原画像(64 × 64 × 64 $voxel$)}
    \label{original_img}
  \end{center}
\end{figure}

\begin{table}[htbp]
  \begin{center}
    \caption{シミュレーション条件}
    \label{simu_cond}
    \small
    \begin{tabular}{|c|c|} \hline
      媒質 & $H_2O$ \\ \hline
      発生光子数 & 200万個/$voxel$ \\ \hline
      初期エネルギー & 140 KeV \\ \hline
      投影数 & 180 投影 \\ \hline
      最大散乱回数 & 5回 \\ \hline
      初期散乱角 & ランダム \\ \hline
      初期方位角 & ランダム \\ \hline
      検出器のサイズ & 512×256×180 $voxel$ \\ \hline
      検出器のピクセルサイズ & 0.08×0.08 $cm^2$ \\ \hline
      カットオフエネルギー & 30 KeV \\ \hline
      画像サイズ & 64×64×64 $voxel$ \\ \hline
      画像のボクセルサイズ & 0.2×0.2×0.2 $cm^3$ \\ \hline
      球の半径 & 5 $cm$ \\ \hline
      回転半径 & 12 $cm$ \\ \hline
      コリメータと検出器間の距離 & 7.5 $cm$ \\ \hline
      ナイフエッジの角度 & 30度 \\ \hline
      ピンホール径 & 0.2 $cm$ \\ \hline
    \end{tabular}
  \end{center}
\end{table}

\newpage
\section{結果}

prymary光子だけを検出した投影画像を図\ref{proj_img}に示す.
\begin{figure}[htbp]
  \begin{center}
    \includegraphics[width=7cm]{./file/proj.png}
    \caption{投影画像(prymary光子)}
    \label{proj_img}
  \end{center}
\end{figure}

また,0度のときの投影画像のプロファイルを図\ref{proj_profile}に示す.
\begin{figure}[htbp]
  \begin{center}
    \includegraphics[width=7cm]{./file/proj_profile.png}
    \caption{0度のときのプロファイル}
    \label{proj_profile}
  \end{center}
\end{figure}

次に,投影画像に感度補正をかけた画像の0度のときの投影画像のプロファイルを図\ref{correction_profile}に示す.
\begin{figure}[htbp]
  \begin{center}
    \includegraphics[width=7cm]{./file/correction_profile.png}
    \caption{感度補正後0度のときのプロファイル}
    \label{correction_profile}
  \end{center}
\end{figure}

最後に,MLEMを100回行った再構成画像とそのプロファイルをそれぞれ図\ref{reconst},図\ref{reconst_profile}に示す.
\begin{figure}[htbp]
  \begin{center}
    \includegraphics[width=3.0cm]{./file/reconst.png}
    \caption{再構成画像(MLEM50回)}
    \label{reconst}
  \end{center}
\end{figure}

\begin{figure}[htbp]
  \begin{center}
    \includegraphics[width=7.0cm]{./file/reconst_profile.png}
    \caption{再構成画像のプロファイル}
    \label{reconst_profile}
  \end{center}
\end{figure}

\section{まとめと今後の展望}
シングルピンホールを用いた3次元画像のモンテカルロシミュレーションを行うことができた.今回は媒質を$H_2O$とした球画像でシミュレーションを行ったが,今後はSheppファントムを用いて複数媒質でシミュレーションを行う.その後,マルチピンホールでモンテカルロシミュレーションを行う.

\end{document}
